\documentclass[a4paper]{article}
\usepackage[T1]{fontenc}
\usepackage[utf8]{inputenc}
\usepackage[italian]{babel}
\begin{document}
\author{Lorenzo Pantieri \and Tommaso Gordini}
\title{Minimalismo}
\maketitle
\tableofcontents
\section{Inizio}
Bene, qui comincia il nostro grazioso articolo\dots
\section{Fine}
\dots e qui finisce.
\end{document}
%• \begin{document} segnala l’inizio del documento;
%• \author e \title (che si possono dare anche prima del comando pre-
%cedente) ne specificano rispettivamente nome dell’autore e titolo;
%• \and si spiega da sé;
%• \maketitle produce il contenuto dei due comandi precedenti, dopo i
%quali deve essere dato;
%• \tableofcontents produce l’indice generale dopo due composizioni;
%• \section{ ⟨ titolo ⟩ } produce un titolo di sezione (un paragrafo, in que-sto caso);
%• \dots produce i puntini di sospensione . . . ;
%• \end{document} segnala la fine del documento.